% Options for packages loaded elsewhere
\PassOptionsToPackage{unicode}{hyperref}
\PassOptionsToPackage{hyphens}{url}
%
\documentclass[
]{article}
\usepackage{lmodern}
\usepackage{amssymb,amsmath}
\usepackage{ifxetex,ifluatex}
\ifnum 0\ifxetex 1\fi\ifluatex 1\fi=0 % if pdftex
  \usepackage[T1]{fontenc}
  \usepackage[utf8]{inputenc}
  \usepackage{textcomp} % provide euro and other symbols
\else % if luatex or xetex
  \usepackage{unicode-math}
  \defaultfontfeatures{Scale=MatchLowercase}
  \defaultfontfeatures[\rmfamily]{Ligatures=TeX,Scale=1}
\fi
% Use upquote if available, for straight quotes in verbatim environments
\IfFileExists{upquote.sty}{\usepackage{upquote}}{}
\IfFileExists{microtype.sty}{% use microtype if available
  \usepackage[]{microtype}
  \UseMicrotypeSet[protrusion]{basicmath} % disable protrusion for tt fonts
}{}
\makeatletter
\@ifundefined{KOMAClassName}{% if non-KOMA class
  \IfFileExists{parskip.sty}{%
    \usepackage{parskip}
  }{% else
    \setlength{\parindent}{0pt}
    \setlength{\parskip}{6pt plus 2pt minus 1pt}}
}{% if KOMA class
  \KOMAoptions{parskip=half}}
\makeatother
\usepackage{xcolor}
\IfFileExists{xurl.sty}{\usepackage{xurl}}{} % add URL line breaks if available
\IfFileExists{bookmark.sty}{\usepackage{bookmark}}{\usepackage{hyperref}}
\hypersetup{
  pdftitle={Biological Oceanography (OCEAN 5051)},
  hidelinks,
  pdfcreator={LaTeX via pandoc}}
\urlstyle{same} % disable monospaced font for URLs
\usepackage[margin=1in]{geometry}
\usepackage{longtable,booktabs}
% Correct order of tables after \paragraph or \subparagraph
\usepackage{etoolbox}
\makeatletter
\patchcmd\longtable{\par}{\if@noskipsec\mbox{}\fi\par}{}{}
\makeatother
% Allow footnotes in longtable head/foot
\IfFileExists{footnotehyper.sty}{\usepackage{footnotehyper}}{\usepackage{footnote}}
\makesavenoteenv{longtable}
\usepackage{graphicx,grffile}
\makeatletter
\def\maxwidth{\ifdim\Gin@nat@width>\linewidth\linewidth\else\Gin@nat@width\fi}
\def\maxheight{\ifdim\Gin@nat@height>\textheight\textheight\else\Gin@nat@height\fi}
\makeatother
% Scale images if necessary, so that they will not overflow the page
% margins by default, and it is still possible to overwrite the defaults
% using explicit options in \includegraphics[width, height, ...]{}
\setkeys{Gin}{width=\maxwidth,height=\maxheight,keepaspectratio}
% Set default figure placement to htbp
\makeatletter
\def\fps@figure{htbp}
\makeatother
\setlength{\emergencystretch}{3em} % prevent overfull lines
\providecommand{\tightlist}{%
  \setlength{\itemsep}{0pt}\setlength{\parskip}{0pt}}
\setcounter{secnumdepth}{-\maxdimen} % remove section numbering

\title{Biological Oceanography (OCEAN 5051)}
\author{}
\date{\vspace{-2.5em}}

\begin{document}
\maketitle

\hypertarget{information}{%
\subsection{Information}\label{information}}

\textbf{Time}: Tuesday 9:10- 12:10

\textbf{Lecturers}: Chih-Hao HSIEH \& Fuh-Kwo SHIAH

\textbf{Credits}: 3

\hypertarget{reference}{%
\subsection{Reference}\label{reference}}

Miller, C. B. (2004) Biological Oceanography. Blackwell Science Ltd,
Oxford, U.K., 402 pp.

\hypertarget{outline}{%
\subsection{Outline}\label{outline}}

This is a course intended for undergraduate and graduate students with
knowledge of basic ecology. The focus will be on OCEANOGRAPHY, with
investigation on interactive biological, chemical, and physical
processes in the ocean. The purposes are to give an overview of
biological ocean science (a wide rather than deep view) and to provide
basic information and training for graduate research. The discussion
will range from physical effects on the biology to biological effects on
biogeochemical cycling; the spatial scale will range from individual
organisms (e.g.~viscosity and turbulences on plankton feeding and
nutrient uptake) to ecosystem (e.g.~remote sensing and circulation
modeling); the organism will range from virus to whales.

\hypertarget{objectives}{%
\subsection{Objectives}\label{objectives}}

The objectives are for students to understand how environmental effects
such as ocean physics and chemistry affect organisms, across temporal
and spatial scales. Further, we will explore how biological activities
feedback to Earth environments, such as biogeochemical cycling and
carbon flux and global climate changes.

\hypertarget{schedule}{%
\subsection{Schedule}\label{schedule}}

\begin{longtable}[]{@{}llll@{}}
\toprule
\begin{minipage}[b]{0.13\columnwidth}\raggedright
Week\strut
\end{minipage} & \begin{minipage}[b]{0.23\columnwidth}\raggedright
Content\strut
\end{minipage} & \begin{minipage}[b]{0.23\columnwidth}\raggedright
Chapter\strut
\end{minipage} & \begin{minipage}[b]{0.30\columnwidth}\raggedright
Lecturers\strut
\end{minipage}\tabularnewline
\midrule
\endhead
\begin{minipage}[t]{0.13\columnwidth}\raggedright
1\strut
\end{minipage} & \begin{minipage}[t]{0.23\columnwidth}\raggedright
Discussion on how to prepare final presentation \& Introduction to
oceanography\strut
\end{minipage} & \begin{minipage}[t]{0.23\columnwidth}\raggedright
NA\strut
\end{minipage} & \begin{minipage}[t]{0.30\columnwidth}\raggedright
Chih-Hao HSIEH \& Fuh-Kwo SHIAH\strut
\end{minipage}\tabularnewline
\begin{minipage}[t]{0.13\columnwidth}\raggedright
2\strut
\end{minipage} & \begin{minipage}[t]{0.23\columnwidth}\raggedright
The major taxa of marine planktoners w/o zooplanktoners\strut
\end{minipage} & \begin{minipage}[t]{0.23\columnwidth}\raggedright
2\strut
\end{minipage} & \begin{minipage}[t]{0.30\columnwidth}\raggedright
Fuh-Kwo SHIAH\strut
\end{minipage}\tabularnewline
\begin{minipage}[t]{0.13\columnwidth}\raggedright
3\strut
\end{minipage} & \begin{minipage}[t]{0.23\columnwidth}\raggedright
Vertical mixing and the spring (phytoplankton) bloom\strut
\end{minipage} & \begin{minipage}[t]{0.23\columnwidth}\raggedright
1\strut
\end{minipage} & \begin{minipage}[t]{0.30\columnwidth}\raggedright
Fuh-Kwo SHIAH\strut
\end{minipage}\tabularnewline
\begin{minipage}[t]{0.13\columnwidth}\raggedright
4\strut
\end{minipage} & \begin{minipage}[t]{0.23\columnwidth}\raggedright
Nutrient cycling: new vs.~regenerated production (f-ratio)\strut
\end{minipage} & \begin{minipage}[t]{0.23\columnwidth}\raggedright
handouts\strut
\end{minipage} & \begin{minipage}[t]{0.30\columnwidth}\raggedright
Fuh-Kwo SHIAH\strut
\end{minipage}\tabularnewline
\begin{minipage}[t]{0.13\columnwidth}\raggedright
5\strut
\end{minipage} & \begin{minipage}[t]{0.23\columnwidth}\raggedright
The ocean carbon cycle and the biological pump\strut
\end{minipage} & \begin{minipage}[t]{0.23\columnwidth}\raggedright
3\strut
\end{minipage} & \begin{minipage}[t]{0.30\columnwidth}\raggedright
Fuh-Kwo SHIAH\strut
\end{minipage}\tabularnewline
\begin{minipage}[t]{0.13\columnwidth}\raggedright
6\strut
\end{minipage} & \begin{minipage}[t]{0.23\columnwidth}\raggedright
Particle sedimentation \& dissolved organic fluxes\strut
\end{minipage} & \begin{minipage}[t]{0.23\columnwidth}\raggedright
handouts\strut
\end{minipage} & \begin{minipage}[t]{0.30\columnwidth}\raggedright
Fuh-Kwo SHIAH\strut
\end{minipage}\tabularnewline
\begin{minipage}[t]{0.13\columnwidth}\raggedright
7\strut
\end{minipage} & \begin{minipage}[t]{0.23\columnwidth}\raggedright
Primary production\strut
\end{minipage} & \begin{minipage}[t]{0.23\columnwidth}\raggedright
handouts\strut
\end{minipage} & \begin{minipage}[t]{0.30\columnwidth}\raggedright
Fuh-Kwo SHIAH\strut
\end{minipage}\tabularnewline
\begin{minipage}[t]{0.13\columnwidth}\raggedright
8\strut
\end{minipage} & \begin{minipage}[t]{0.23\columnwidth}\raggedright
Midterm\strut
\end{minipage} & \begin{minipage}[t]{0.23\columnwidth}\raggedright
NA\strut
\end{minipage} & \begin{minipage}[t]{0.30\columnwidth}\raggedright
NA\strut
\end{minipage}\tabularnewline
\begin{minipage}[t]{0.13\columnwidth}\raggedright
9\strut
\end{minipage} & \begin{minipage}[t]{0.23\columnwidth}\raggedright
Marine zooplankton and their life styles\strut
\end{minipage} & \begin{minipage}[t]{0.23\columnwidth}\raggedright
handouts\strut
\end{minipage} & \begin{minipage}[t]{0.30\columnwidth}\raggedright
Chih-Hao HSIEH\strut
\end{minipage}\tabularnewline
\begin{minipage}[t]{0.13\columnwidth}\raggedright
10\strut
\end{minipage} & \begin{minipage}[t]{0.23\columnwidth}\raggedright
Ocean instruments\strut
\end{minipage} & \begin{minipage}[t]{0.23\columnwidth}\raggedright
handouts\strut
\end{minipage} & \begin{minipage}[t]{0.30\columnwidth}\raggedright
Chih-Hao HSIEH\strut
\end{minipage}\tabularnewline
\begin{minipage}[t]{0.13\columnwidth}\raggedright
11\strut
\end{minipage} & \begin{minipage}[t]{0.23\columnwidth}\raggedright
Foodweb dynamics and microbial loop\strut
\end{minipage} & \begin{minipage}[t]{0.23\columnwidth}\raggedright
5\strut
\end{minipage} & \begin{minipage}[t]{0.30\columnwidth}\raggedright
Chih-Hao HSIEH\strut
\end{minipage}\tabularnewline
\begin{minipage}[t]{0.13\columnwidth}\raggedright
12\strut
\end{minipage} & \begin{minipage}[t]{0.23\columnwidth}\raggedright
Remote sensing\strut
\end{minipage} & \begin{minipage}[t]{0.23\columnwidth}\raggedright
6\strut
\end{minipage} & \begin{minipage}[t]{0.30\columnwidth}\raggedright
Chih-Hao HSIEH\strut
\end{minipage}\tabularnewline
\begin{minipage}[t]{0.13\columnwidth}\raggedright
13\strut
\end{minipage} & \begin{minipage}[t]{0.23\columnwidth}\raggedright
Secondary production\strut
\end{minipage} & \begin{minipage}[t]{0.23\columnwidth}\raggedright
7\strut
\end{minipage} & \begin{minipage}[t]{0.30\columnwidth}\raggedright
Chih-Hao HSIEH\strut
\end{minipage}\tabularnewline
\begin{minipage}[t]{0.13\columnwidth}\raggedright
14\strut
\end{minipage} & \begin{minipage}[t]{0.23\columnwidth}\raggedright
Population dynamics\strut
\end{minipage} & \begin{minipage}[t]{0.23\columnwidth}\raggedright
8\strut
\end{minipage} & \begin{minipage}[t]{0.30\columnwidth}\raggedright
Chih-Hao HSIEH\strut
\end{minipage}\tabularnewline
\begin{minipage}[t]{0.13\columnwidth}\raggedright
15\strut
\end{minipage} & \begin{minipage}[t]{0.23\columnwidth}\raggedright
Coupled biological/physical modeling\strut
\end{minipage} & \begin{minipage}[t]{0.23\columnwidth}\raggedright
4\strut
\end{minipage} & \begin{minipage}[t]{0.30\columnwidth}\raggedright
Chih-Hao HSIEH\strut
\end{minipage}\tabularnewline
\begin{minipage}[t]{0.13\columnwidth}\raggedright
16\strut
\end{minipage} & \begin{minipage}[t]{0.23\columnwidth}\raggedright
Fisheries oceanography\strut
\end{minipage} & \begin{minipage}[t]{0.23\columnwidth}\raggedright
15\strut
\end{minipage} & \begin{minipage}[t]{0.30\columnwidth}\raggedright
Chih-Hao HSIEH\strut
\end{minipage}\tabularnewline
\begin{minipage}[t]{0.13\columnwidth}\raggedright
17\strut
\end{minipage} & \begin{minipage}[t]{0.23\columnwidth}\raggedright
Global climate change\strut
\end{minipage} & \begin{minipage}[t]{0.23\columnwidth}\raggedright
16\strut
\end{minipage} & \begin{minipage}[t]{0.30\columnwidth}\raggedright
Chih-Hao HSIEH\strut
\end{minipage}\tabularnewline
\begin{minipage}[t]{0.13\columnwidth}\raggedright
18\strut
\end{minipage} & \begin{minipage}[t]{0.23\columnwidth}\raggedright
Final\strut
\end{minipage} & \begin{minipage}[t]{0.23\columnwidth}\raggedright
NA\strut
\end{minipage} & \begin{minipage}[t]{0.30\columnwidth}\raggedright
Chih-Hao HSIEH\strut
\end{minipage}\tabularnewline
\bottomrule
\end{longtable}

\hypertarget{evaluation} Midterm \textbf{30\%} Final \textbf{40\%} Oral
presentation (a list of research topics named ``Marine Ecology in the
Cutting Edge'' will be provided. Biological oceanography is a
multidisciplinary science, and we need to learn what other fields can
help us to understand our ocean. For example, how engineer help design
sampling devices that are useful in collecting biological data. Each
student is required to pick a topic from the list and give a
presentation on that topic.Suitable references will be provided)

\end{document}
